\section*{Resumé}
We were a four-man group which worked on the Giraf project, which is a collection of apps that helps children with autism and their caretakers. The aim of the project this semester, as it entered its fifth year of development, was to consolidate earlier gains; only a few new features were added and there was a focus on making the project ready for deployment.

Our group worked in the subproject called Build and Deployment (B\&D), which is a division of the greater whole of the project focused entirely on making the deployment as smooth as possible and easing the workload for the other developers. We had roles as product owners, and were responsible for managing Google Play and Google Analytics.

As a B\&D group we worked on improving the development environment through AAR files. We wrote guides to be shared on Redmine for how to use Google Analytics, how to forward crash reports from Google Analytics, how to use Google Play, how to set up Android Studio, how to rename apps and libraries in Android Studio, how to implement the pictosearch library, and how to use Doxygen for code documentation purposes. In addition to this, we mapped the dependencies of the apps, which was important for our efforts in making the apps standalone. We also changed an app into a library, as a part of a process to make the apps standalone.

As product owners we were a driving force in the organization of sprints. We chaired the sprint planning and the sprint ends for the second, third, and fourth sprint for our subproject. We oversaw the assignment of user stories, which we wrote ourselves upon request and kept up to date throughout the project period. In cooperation with the other product owners we worked out a common standard for the user stories and the backlog and evaluated each other’s work together.

As responsible for Google Play and Google Analytics we worked on autoforwarding crash reports and on making the apps send in the reports for us to see in the first place. We added an alpha and beta track to the version control on Google Play. We spearheaded the renaming efforts across the project to ensure consistency in the naming of the apps from their titles to their package-names and repositories.