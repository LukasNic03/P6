\subsection{Google Play}
Google Play is a digital distribution platform created by Google. It is used as an appstore for android devices as well as for music and e-book distribution. Because we in the project are making an apps for android devices, we are using the Google Play store to distribute the different apps. 

As it was our responsibility to maintain and control Google Play for this semester, we started by familiarize us with the different features that Google Play provides. We found that Google Play provides a lot of statistics for all the apps, like how many times the app has been installed and how many are still using it. Like having access to statistics, Google Play also allowed os to change with apps should be shown on the store and to change the descriptions for the apps. One thing we noticed when we looked at some of the description was that a lot of the descriptions was very short and did not provide enough information about the apps for the user to know what functionality the apps provided. Because of this we decided to update some of the descriptions, make an English version, and to update some of the pictures provided for the user.

One data that Google Play provided ,that was very important, was the statistic about with version of android the users where using. because from that we found that the apps has to use the android API 15, because 22\% of the users where using android version 4.0.3 - 4.0.4, with can only run with API 15 or below. 

\subsubsection{Alpha/Beta}
As an end to the last years progress the apps from the project was all uploaded to the Google play store. This meant that the apps became available for the users, and that data about crashes could be send to Google play and analytics. The version on the Google play store was only a released version, but for the sake of expanding the use of the Google store it was decided to include the use of alpha and beta build for the apps. The versions was then planned to represent different stages in the builds. Released version should be finished and stable apps, beta should be stable apps, and alpha, apps that are being worked on right now. This split between version ensured that the users only get the finished and most stable versions of the apps, and at the same time allow the developers to test the unfinished alpha and beta version on the android device, and by that still get crash reports from Google play and analytics for those apps.

For the work in this implementation of alpha and beta versions, we found a plug-in for Jenkins, that could be used for making the uploading of alpha and beta version automatically, as well as providing the information the server needed to upload to Google Play. The plug-in and information was then given to the group that controlled the Jenkins server, so that they could implement it. 