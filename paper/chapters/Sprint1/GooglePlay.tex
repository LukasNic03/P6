\subsection{Google Play}
Google Play is a digital distribution platform created by Google. It is used as an appstore for android devices as well as for music and e-book distribution. Because we in the project are making an apps for android devices, we are using the Google Play store to distribute the different apps. 

As it was our responsibility to maintain and control Google Play for this semester, we started by familiarize us with the different features that Google Play provides. We found that Google Play provides a lot of statistics for all the apps, like how many times the app has been installed and how many are still using it. Like having access to statistics, Google Play also allowed os to change with apps should be shown on the store and to change the descriptions for the apps. One thing we noticed when we looked at some of the description was that a lot of the descriptions was very short and did not provide enough information about the apps for the user to know what functionality the apps provided. Because of this we decided to update some of the descriptions, make an English version, and to update some of the pictures provided for the user.

One data that Google Play provided with was very important was the statistic about with version of android the users where using. because from that we found that the apps has to use API 15, because 22\% of the users where using android version 4.0.3 - 4.0.4. 

\subsubsection{Alpha/Beta}
One thing that the members of build and deploy found that where missing in the Google Play store was the use of alpha and beta builds. As it wa when we started the semester there only was one version of all the app that where released, but for the sake of being able to easier test the apps without releasing possible unstable versions to the user, it was decided to make alpha and beta version of the apps, that were only available for the Giraf project members. The alpha version was then versions of the apps that were released every time someone updates the app, and the beta version was set to be released once every week. 

For the work in this implementation of alpha and beta versions, we found a plug-in for Jenkins, that could be used for making the uploading of alpha and beta version automatically, as well as providing the information the server needed to upload to Google Play. The plug-in and information was then given to the group that controlled the Jenkins server, so that they could implement it. 