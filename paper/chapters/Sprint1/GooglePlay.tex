\subsection{Google Play}
Google Play is a digital distribution platform created by Google. It is used as an appstore for Android devices as well as for music and ebook distribution. Because the goal of the project is to make apps for android devices, we are using the Google Play store to distribute the different apps. \citep{GooglePlay}

As it was our responsibility to maintain and control Google Play for this semester, we started by familiarizing ourselves with the different features provided by Google Play. We found that Google Play provides a lot of statistics for all the apps, like how many times the apps have been installed and how many are still using them. Google Play also allowed os to change which apps should be shown on the store and to change the descriptions of the apps. One thing we noticed when we looked at some of the descriptions was that a lot of the descriptions were very short and did not provide enough information about the apps, so the user would have trouble knowing the functionalities that the apps provide. Because of this, we decided to update the descriptions, make an English version of the descriptions, and to update some of the pictures showing the apps to the user.

One data that Google Play provided ,that was very important, was the statistic about with version of android the users where using. because from that we found that the apps has to use the Android API 15, because 22\% of the users where using android version 4.0.3 - 4.0.4, which only supports API 15 and below \citep{API15}. 

\subsubsection{Alpha/Beta}
At the end of last year’s progress, the apps from the project were all uploaded to Google Play. This meant that the apps became available for the users, and that data about crashes could be sent to Google Play and Google Analytics. The version on Google Play was only a released version, but for the sake of expanding the use of Google Play, it was decided to include the use of alpha and beta builds for the apps. The versions were then planned to represent different stages in the builds. Released versions should be finished and stable apps, beta should be stable apps, and alpha versions should be apps that are being worked on right now. This differentiation between versions, ensures that the users only get the finished and most stable versions of the apps, while it also allows the developers to test the unfinished alpha and beta versions on the android devices, and thereby still get crash reports from Google Play and Google Analytics for those versions of the apps.

For the work in this implementation of alpha and beta versions, we found a plug-in for Jenkins that could be used to make the uploading of alpha and beta versions automatic, as well as providing the information the server needs to upload to Google Play. The plug-in and information was then given to the group that controlled the Jenkins server, so they could implement it. 