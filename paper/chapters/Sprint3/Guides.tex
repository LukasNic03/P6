\section{Improvement of wiki pages}
\textit{In this section is a description of how we in the 3rd sprint worked on improving the quality of the guides and information pages on the project wiki pages.}

\subsection{Motivation}
Improving the quality of the information shared on the redmine wiki, like guides and summaries, would improve the workflow by shortening the time a groups uses on understanding the software configuration management tools used in the project, by providing useful guides and guidelines.


\subsection{Process}
Process
As the first step we looked through the whole wiki and listed all the guides and pages that provided information. We then looked through the list and discussed in the group about, if there were any guides that were not on the list that needed to be included. In this discussion we did not find any extra guides that we thought were needed on the wiki page. All the guides and information pages we found are shown on the list below:

\begin{itemize}
	\item Git guide
	\item Android guide and guidelines
	\item Testing
	\item Issue Guidelines
	\item Google Analytics
	\item Javadocs 
	\item Dependencies Management
	\item Guide for renaming Apps and libraries in android studio
	\item Continuous Integration Guidelines
	\item App and library renaming in Android Studio
	\item How to implement the Pictosearch library
	\item Process
	\item Repository names
	\item Dependencies between apps and libs
	\item Database architecture
	\item Requirements Specification and user stories for Giraf
	\item Requirements for Database Component
	\item Links
	\item Meetings
	\item Backlog sprint \#4 last year projects
	\item Product backlog
	\item A list of groups and their responsibilities
	\item Word list, for terminology when talking/writing about the institutes.
	
\end{itemize}

After we had discussed if any guides were missing, we looked through all the guides again, but this time we checked if the content and the formating of the page were correct and of high enough quality. For the pages that did not provide good enough content, but where we had the knowledge ourselves to correct, we corrected ourselves, and for the one we did not have the knowledge to fix we contacted the group that had written it in the first place. For content there were only two instances of pages that did not live up to the standard, the first was “Dependencies management”, where information about how to log in to the maven repository was missing, and the second was the jenkins guide which only had titles. The login information we inserted to the page ourselves. but for the jenkins page we had to contact the group responsible for jenkins in the project. It turned out that the wiki page was not supposed to have been made in the first place, so it was decided that the page got removed, instead of writing the whole guide, Because only the jenkins group would have use of the information, and they already had the knowledge themselves.

A lot of the pages did not have the correct formatting or did not have a formatting at all. For those pages we corrected the formatting so that the pages was easier to read. In addition to the formating, some of the pages also had most of their content in a PDF file. For those pages we wrote the content of the PDF file into the page and then placed a link at the top of the page where the PDF could be downloaded.

For some of the pages we also changed the title so that it was easier to understand what content was on the page without having to open the page.
After having fixed all the guides and information pages on the wiki we send an email to all groups informing that we had fixed all the guided and a list of what guides was available at the moment.

