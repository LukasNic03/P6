\section{Android Studio Setup}

\begin{enumerate}
	\item Download and intall git from http://git-scm.com/
	\item Download Android Studio 1.0.1 and install it http://tools.android.com/download/studio/canary/1-0-1
	\begin{enumerate}
		\item If you do not have Java JDK get it here http://www.oracle.com/technetwork/java/javase/downloads/index.html
	\end{enumerate}
	\item Download Stand alone SDK here http://developer.android.com/sdk/installing/index.html?pkg=tools
	\item Open Android SDK Manager
	\item Install these packages
	\begin{enumerate}
		\item Android SDK Tools
		\item Android SDK Platform-tools
		\item Android SDK Build-tools ver 19.0.1
		\item Android API 15, 17 og 19
		\item Android Support Repository
		\item Android Support Library
		\item Google USB Driver
	\end{enumerate}
	\item Dowloand gradle 2.2.1 here https://services.gradle.org/distributions/gradle-2.2.1-all.zip
	\item Go to http://cs-cust06-int.cs.aau.dk/git, to see the list of projects you have access to, using your student login
	\item In Android Studio, choose Check out from Version Control, using Git with path: http://cs-cust06-int.cs.aau.dk/git/{project-name}
	\item Download a project thourgh AS using option git.
	\item A good start could be the launcher
	\item Point Projects Gradle to your downloaded gradle 2.2.1 folder, using local distribution Point Android Studio to installed SDK
	\item In the project folder run (using CMD or Git Bash)
	\begin{enumerate}
		\item git pull
	\end{enumerate}
	\item From here you should be able to complete the import
	\item The first time Android Studio start up the project, there could be some first-time processing, which could take quite a long time, like indexing. This will be a one-time only processing.
	\item Try to compile a project
\end{enumerate}

\textbf{Troubleshooting; If you get error:}
bad class file magic (cafebabe) or version (0034.0000)
Solution: Make Sure you use Java JDK 7