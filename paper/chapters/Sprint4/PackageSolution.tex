\section{Package Solution} \label{Sprint4_PackageSolution}
\textit{The customers requested a simple and easy way to install and run every app in Giraf. This section will cover our approach to the problem, and we present a possible solution.}

\subsection{Motivation}
As the target audience are not technically savvy, we should make the download and installation process for the apps as simple and intuitive as possible. One thing that the customers specifically requested was the ability to download and install the entire Giraf-system by just downloading one app. This would also make it easier to use the system for people new to the system as they would not have to download several separate apps to use it.

\subsection{Approach}
The first thing we did was to research whether Google Play had any incorporated methods to combine apps or at least download and install multiple apps from the same Google Play store-page. From our research we found that Google have not made anything that can accomplish this.\\
As Google Play does not have the functionality to do what we need, we have to combine the apps ourselves. We present two approaches, one would be to code a single app, which would then have the functionality of all current and possible future apps. The other approach would be to modify the current launcher and add the other apps as libraries in a similar manner as to how pictosearch is used.
The first approach would make it much harder to split up the work to multiple groups, which would go against the Scrum process used in the project. Using the second approach would solve this problem and as such that would be our approach to the problem.

During a semester meeting with the other groups, it was decided that making a package solution where every app were combined to one, was not a high priority, and it would take more time then there were left for this sprint. The consequence and approaches to having one complete app which include all the apps, will be taken up in Section \ref{nextyear_package}.

